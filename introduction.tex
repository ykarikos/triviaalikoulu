\chapter*{Esipuhe}

Porin Triviaalikoulun nuottikirjat vuodelta 1725 ovat lähes täydellisesti säilynyt neljän stemma- eli äänikirjan sarja, joka edustaa monipuolisesti Suomenkin kouluissa ja kirkoissa käytettyä Itämeren alueen moniäänistä koulumusiikkiohjelmistoa. Porin triviaalikoulun nuottikirjat on Graduale Aboensen, \textit{Piae Cantiones} -julkaisujen ja Heinolan pataljoonan nuottikirjojen veroinen Suomen musiikinhistorian monumentti, joka sekä historiallisten että musiikillisten ansioittensa puolesta ansaitsee tulla nykyistä paremmin tunnetuksi. Nuottikirjojen musiikki on käyttökelpoista tänäkin päivänä kirkoissa, kouluissa ja konserteissa.

Imatralainen vanhan musiikin yhtye Sonus Borealis on pitänyt nuottikirjoihin sisältyvää musiikkia ohjelmistossaan jo kahdenkymmenen vuoden ajan. Alkuperäisten nuottikirjojen täyttäessä 300 vuotta on korkea aika julkistaa kokoelma näissä konserteissa koeteltuja sävellyksiä muodossa, joka on kaikenlaisen musiikin harrastajien ja ammattilaisten käytettävissä: modernina partituurimuotoisena laitoksena, joka on tehty esittäjien, ei tutkijoiden tarkoituksia silmälläpitäen.

Vaikka kyseessä ei ole tieteellinen ns. kriittinen laitos, on nuottien yhteyteen kuitenkin valmistettu tämä esipuhe silmälläpitäen käyttäjiä.

\includegraphics[scale=0.35]{../facsimile/christus-adest-justus}

\section*{Nuottikirjojen tausta ja sisältö}

Porin triviaalikoulun nuottikirjat sisältävät koulumusiikkia, ja on ensiksi syytä selvittää, mitä tällä tarkoitetaan. 1500–1700-lukujen koulumusiikki ei ole sama ilmiö kuin keskiajalta polveutuvat niin sanotut teini- eli opiskelijalaulut. Viime mainittua perinnettä edustaa hyvin tunnettu \textit{Piae Cantiones} -julkaisu, jonka myöhempään 1625 painokseen kuitenkin sisältyy ajankohtaisena päivityksenä myös uutta, Daniel Fridericin säveltämää koulumusiikkia. Porin Triviaalikoulun nuottikirjat, ja niiden kanssa osin yhtenevät mutta suppeammat Vaasan Triviaalikoulun nuottikirjat, sisältävät moniäänistä reformaation jälkeisen ajan koulumusiikkia: alkuperältään uudempaa, mutta ei uusinta uutta polyfoniaa, joka on tarkoitushakuisen yksinkertaista siten, että se soveltuu musiikin opetukseen yhtä hyvin kuin ammattikäyttöön.

Musiikki oli keskeinen osa luterilaisten maiden koulusivistystä 1700-luvulle asti ja vähemmässä määrin pitkään tämän jälkeenkin. Itämeren luterilaisissa maissa oli 1600–1700-luvuilla erityinen, melko yhtenevä koulumusiikkiohjelmisto, jonka kivijalka oli 1500-luvun ja 1600-luvun alun vokaalipolyfonia. Tämä vanhakantaisuus ei ollut osoitus Pohjolan taantumuksellisuudesta, sillä pitkälti sama ohjelmisto päti myös Pohjois-Saksan varakkaisiin ja hyvin verkostoituneisiin musiikkikeskuksiin. Itse Johann Sebastian Bach tavallisesti aloitti jumalanpalvelukset koulumusiikkityyliin vanhahtavalla, kirkkovuoden ajankohdan mukaan valitulla kaksoiskuoromotetilla. Tätä tarkoitusta varten hän itse hankki 1729 koulunsa käyttöön klassikon aseman saavuttaneen \textit{Florilegium Portense} -kokoelman, joka tuolloin oli jo yli sata vuotta vanha. Muutamaa vuotta aiemmin tuosta samasta kokoelmasta peräisin olevia kaksoiskuoromotetteja kopioitiin käsin porilaisten koululaisten käyttöä varten.

Porin triviaalikoulun nuottikirjat koostuvat neljästä äänikirjasta, jotka on kopioitu käsin vuonna 1725 triviaalikoulun käyttöön, todennäköisesti korvaamaan isovihan aikana hävitettyjä aiempia nuottikirjoja. Kokoelmaan sisältyvät sävellykset ovat 4–8-äänisiä, valtaosaltaan latinankielistä kirkkomusiikkia, kuten koulumusiikki yleensäkin. Sävellysten kokonaismäärä on noin yhdeksänkymmentä. Epätarkka arvio johtuu siitä, että sävellysten määrä vaihtelee eri äänikirjojen sisällysluetteloissa ja riippuen eri kieliversioiden laskentatavasta. Kirjojen kopiointityö näyttää tuntemattomasta syystä jääneen osittain suunnitelmavaiheeseen.

Yksi nykyajan näkökulmasta huomattava puute Porin käsikirjoituskirjoissa on säveltäjänimien puuttuminen. Vaikuttaa ilmeiseltä, että säveltäjätiedot eivät kiinnostaneet kirjojen kopioijaa ja käyttäjiä. Rinnakkaisten lähteitten perusteella osalle sävellyksistä on ollut mahdollista jäljittää säveltäjä. Jäljittämistä vaikeuttaa se, että osa on uudelleentekstitetty.

Porin triviaalikoulun nuottikirjat löysi Porin Lyseon kirjaston kätköistä lehtori Eeli Granit-Ilmoniemi, ja niitä teki tunnetuksi Heikki Klemetti, joka myös esitti ja sovitti niissä olevia sävellyksiä. Tämä tapahtui yli sata vuotta sitten, ja sen jälkeen nuottikirjat ovat saaneet levätä rauhassa lukuunottamatta Gudrun Viergutzin Jyväskylässä tekemää arvokasta taustatutkimus- ja digitointityötä. Vuonna 2005 Johannes Vesterinen kiinnostui kirjoista ja alkoi vähin erin kirjoittaa sävellyksiä puhtaaksi Sonus Borealis -yhtyeen esityksiä varten.

\section*{Käyttökonteksti}

Nykyisenkaltainen sekulaari konserttikulttuuri oli 1700-luvulla lapsenkengissään, eikä sellaista Suomessa ollut lainkaan. Nuottikirjojen musiikin sisältö on valtaosaltaan hengellistä, ja musiikin pääasiallinen käyttötarkoitus luterilaisen ortodoksian aikana, ennen valistusajan sekularisaatiota ja jumalanpalvelusten rappeutumista luentotilaisuuksiksi, lienee ollut liturginen. Osalle teksteistä on selkeä konteksti vanhamuotoisen kirkkovuoden ajankohdan mukaan. Joukossa on myös muutama muuhun juhlakäyttöön sopiva maallinen sävellys, mutta yhtäkään niistä ei osunut mukaan valikoimaamme.

Äänikirjojen hyvä kunto ja korjausmerkintöjen vähäisyys antaa vaikutelman, että kirjoja on käytetty Porissa harvakseltaan. Vaikka nuottikirjojen säilytyspaikka on ollut koulu, vaikuttaa todennäköiseltä, että niitä on käytetty vain merkittävinä juhlapäivinä, eikä siis opetustarkoituksessa. Näissä tilanteissa todellinen esityskokoonpano on saatettu koostaa koulun etevimmistä oppilaista yhdistettynä muihin kynnellekykeneviin laulajiin ja soittajiin. Turusta on säilynyt kirjallisia tietoja tällaisista, isompia juhlallisuuksia varten kirkkojen, koulujen, kaupunkien ja sotajoukkojen musikanteista kootuista yhteisvoimista, jotka ovat mahdollistaneet jopa monikuoroisen musiikin esittäminen. Porissa on kenties toimittu samoin, mutta pienemmässä mittakaavassa.

\section*{Esitystapa}

Porin Triviaalikoulun musiikin nuottikuva antaa nykylukijalle vaikutelman sekakuoromusiikista, mutta tämä on harhakuva. Soitinten käyttö vanhaan tyyliin (stile antico) sävelletyssä vokaalimusiikissa oli 1600- ja 1700-lukujen yleinen tapa. Tämän jo Klemetti tiesi. Sitkeässä on kuitenkin yhä ajatus, että vokaalipolyfonia olisi historiallisesti tai ainakin ideaaliselta periluonteeltaan a capella -sekakuoromusiikkia, jota soitinkaksinnus korkeintaan puujalkamaisesti tukee. Aikalaiskuvauksista ja muun muassa Michael Praetoriuksen (1571–1621) julkaisuista avautuu hyvin toisenlainen kuva polyfonisen musiikin tekemisestä: soittimet kuuluivat olennaisesti asiaan, ja niitä käytettiin hyvällä harkinnalla, mutta hyvin vapaasti ja monipuolisesti.

Sonus Borealis -yhtyeen konserteissa ja äänilevyllä on seurattu mahdollisimman monipuolisin tavoin 1600-luvun esitystapaa käyttäen sooloääniä ja pientä ripienokuoroa sekä melko runsasta valikoimaa soittimia. Tämä tarkoittaa standardin \textit{basso continuo} tai \textit{bassus pro organo} -tyyppisen urku- tai regaalisäestyksen lisäksi, tai sijasta, lauluäänien kaksintamista tai korvaamista melodiasoittimin. Historiallisesti tavallisinta oli sinkkien, pasuunoiden ja jousisoitinten käyttö. Nykyisin ovat puhaltimista nokkahuilut helpoiten saavutettavissa. Skalmeijat ja pommerit olivat yleisesti kaupunginmuusikoiden käytössä 1600-luvulla, ja 1700-luvulla saapuivat sotilassoittajien mukana Ranskassa kehitetyt oboet ja fagotit. Näitä kaikkia on saatettu käyttää koulumusiikkiohjelmiston esittämiseen, joskin on mahdotonta sanoa, miten juuri Porissa on toimittu. Porihan oli vain yksi useista Suomen koulukaupungeista ja on onnekas sattuma, että juuri sieltä on säilynyt yksi kattava nuottikirjasarja, muiden tuhoutuessa tulen, sodan ja välinpitämättömyyden seurauksena.

Urkuja oli Suomessa vähän, mutta missä niitä oli, saattoi niiden äänikerroilla korvata monenlaisia soittimia. Kyseessä oli aina pieni tai iso kirkkourku. Nykyisten vanhan musiikin yhtyeiden rutiininomaisesti käyttämät kuljetettavat continuopositiivit eivät ole historiallisia, mutta niiden kuljetettavuus ja viritettävyys tekevät niistä nykyisin lähes vastustamattomia.

Kaksoiskuoromotetit ovat nuottikirjojen loistokkainta musiikkia. Jälleen on syytä avata käsitteistöä: historiallinen kaksoiskuoro ei tarkoita suurkuoroa. "Kuoro" 1600-luvun kielenkäytössä tarkoitti esittäjäkokoonpanoa yleisesti, täysin riippumatta siitä, oliko kussakin stemmassa yksi vai useampia esittäjiä, ja myös riippumatta siitä, olivatko kyseessä laulajat vai soittajat. Johann Sebastian Bachin kaksoiskuoromoteteissa näkyy myöhäinen käytäntö kaksintaa laulajat soittimilla. Varhaisemmissa lähteissä kuvataan tavallisena myös esitystapaa, jossa kussakin kuorossa on vain yksi ääni laulettuna ja loput soitettuna niin, että kuoroissa on toisiaan kontrastoiva sointi. Michael Praetorius esittelee \textit{Syntagma Musicum} -teoksensa kolmannessa osassa näitä ja muita soitinnustapoja, ja myös mahdollisuuden käyttää eri tavoin rekisteröityjen urkujen sormioiden vuorotteluja monikuoroefektien toteuttamiseen.

Kuorot oli tavallisesti erotettu fyysisesti erilleen toisistaan siten, että syntyy stereofoninen kuorojen välinen vuoropuhelu. Kuorojen sijoittelu erilleen juontaa juurensa antifonisesta psalmilaulannasta vanhoilla gregoriaanisilla psalmisävelmillä, ja tämän käytännön perintönä se oli 1600-luvulla tavallista, ja koski jopa 4- tai 5-äänistä yksikuoromusiikkia kaksintavia ripienokuoroja (joita nykyään kutsuttaisiin "tutti-kuoroiksi"). Tässä kohtaa on iso kynnys nykymuusikoille ja harrastajille, jotka ovat tottuneet toimimaan yhtenä tiiviinä ryhmänä yleisönsä edessä: haasteena on, että mitä isommaksi kuorojen välinen etäisyys tulee, sitä huonommin kuorot kuulevat toisiaan, ja on seurattava johtajaa. Mihin kyettiin 1500-1600-luvuilla, sen ei pitäisi olla mahdotonta 2000-luvullakaan, ja haluamme rohkaista uuden nuottikirjan käyttäjiä paitsi soitinten myös monikuoroisuuden rohkeampaan käyttöön.

Isojen sekakuorojen asema nykyajan kuoromusiikissa on niin itsestäänselvä, että on vaikea kuvitella tämän standardin vakiintuneen vasta 1700-luvun lopulla. Mikään ei tietenkään estä laulamasta musiikkia kuinka suurella kuorolla hyvänsä, mutta on hyvä tiedostaa, että nykyisenkaltaista kirkkokuorokäytäntöä ei vielä 1700-luvun alkupuolella tunnettu. Itämeren maissa vielä J. S. Bachin aikoihin asti varsin tavallista oli yksi laulaja per ääni. Näin syntyvästä soinnista puuttuu sekakuoron massiivisuus, mutta tilalle avautuu mahdollisuuksia kamarimusiikilliseen ketteryyteen ja spontaaniin, improvisoituun koristeluun. Toisaalta tältäkin pohjalta tiedetään suureellisimmissa tilanteissa rakennetun, lukuisien kaksintavien soitin- ja laulajakuorojen kautta, suurkokoonpanoja, joiden sointivaikutelma on varmasti ollut paitsi muhkean värikäs, myös massiivinen.

Mitä laulajiin tulee, Porin triviaalikoulun nuottikirjasta on syytä huomioida, että vokaalimusiikin perusäänialat olivat 1500–1600-luvuilla matalammat kuin nykyisin, ja että vanhan ohjelmistonsa puolesta vuoden 1725 kirjat edelleen noudattavat tätä vanhaa äänialajakoa:

\begin{table}[h!]
\begin{tabular}{l l l}
\multicolumn{2}{l}{\textbf{Historiallinen ääniala}} & \textbf{Nykyinen vastinkäsite} \\
\hline
Discantus	& (b–)c'–e''(–f’’) & mezzosopraano (lapsi- tai naisääni tai kontratenori) \\
Altus		& (d–)f–a'	& (korkea) tenori (joskus matala lapsi- tai naisaltto) \\
Tenor		& (A–)c–e'	& baritoni \\
Bassus	    & (E–)F–a	& basso \\
\end{tabular}
\end{table}

Nykyisenkaltaista yhteisesti sovittua, absoluuttista viritysstandardia ei 1600-1700-luvuilla ollut. Puhallinsoittimet ovat kuitenkin olleet virityskorkeuden kiintopiste. Säilyneistä soittimista tiedetään, että urkujen, sinkkien ja pasuunoiden viritystaso oli Pohjois-Euroopassa tavallisesti noin kokoaskelen nykyistä korkeampi. Laulajia varten oli kuitenkin tavallista transponoida tästä kokosävelaskelta alemmaksi, kuten Praetorus neuvoo. Modernilla SATB-sekakuorolla miellyttävämpiin äänialoihin päästään transponoinnilla kokoaskel ylöspäin.

Mitä transponointin tulee, kokonaan oma kysymyksensä on, että osa Porin triviaalikoulun kappaleista on kirjoitettu 1500-luvulla yleisen käytännön mukaisesti niin sanotuille korkeille avaimille (chiavi alti, chiavetti). Niitä oli eri yhdistelmiä, mutta ne aina tunnistaa siitä, että alin ääni oli kirjoitettu muulle kuin bassoavaimelle siten, että se näyttää kulkevan tenorin äänialassa. Tästä syntyy illuusio, että osa kappaleista on noin kvinttiä muita korkeammassa äänialassa, mutta tämä on todellakin illuusio. Korkeille avaimille kirjoitetut kappaleet transponoitiin rutiininomaisesti kvartilla tai kvintillä alaspäin, jolloin niiden soiva ääniala oli sama tai hivenen matalampi kuin normaaleilla avaimilla kirjoitetussa musiikissa. Valmistamassamme uudessa nuottieditoissa transponointi on tehty valmiiksi, joten nuottien käyttäjä voi laulaa tai soittaa suoraan nuotista vaivaamatta transponointikysymyksillä päätään.

Nykyaika tarjoaa monia mahdollisuuksia Porin triviaalikoulun musiikin esittämiseen eri tavoin. Tiukan historiallisen autenttisuuden periaatteiden seuraajat saattavat pyrkiä rekonstruoimaan hypoteettisia Porin esitysolosuhteita. Historiallisesti aivan yhtä perusteltua on tarkastella koulumusiikkia sen laajemmassa kontekstissa, ja Sonus Borealiksen tapaan vapaasti hyödyntää kaikkia niitä historiallisia esitystapojen mahdollisuuksia, mitä koulumusiikkiohjelmistoon on koko Itämeren alueella liittynyt. Oman kokemuksemme mukaan viime mainittu lähestymistapa tekee musiikille eniten oikeutta ja samalla tarjoaa esittäjille ja kuulijoille monipuolisemman kuuntelukokemuksen, etenkin konserteissa.

Mikään ei tietenkään estä viemästä sovitustyötä kokonaan ulos historiallisista puitteista, kuten Heikki Klemetti teki sovittaessaan Jacobus Galluksen Heroes pugnaten marssiksi sotilassoittokuntien käyttöön. Vapaammasta soveltamisesta esimerkkinä on julkaisumme viimeisenä nuottina oleva Johannes Vesterisen sovitus neliäänisestä kappaleesta Veni Sancte Spiritus. Se on eräänlainen tarjoiluehdotus mahdollisuuksista, joita juuri kokoelman yksinkertaisimpien ja lyhyimpien kappaleiden luovaan toteuttamiseen eri musiikkiperinteitä soveltamalla avautuu.

\section*{Uusi Porin Triviaalikoulun nuottikirja}

Käsillä oleva uusi editio Porin Triviaalikoulun nuottikirjoista, lajissaan ensimmäinen, on 29 kappaleen valikoima niistä sävellyksistä, joita soitin- ja lauluyhtye Sonus Borealis on konserteissaan esittänyt. Suuri osa niistä löytyy myös yhtyeen levyltä \textit{Musiikkia Porin Triviaalikoulun nuottikirjoista} (2008). Sonus Borealista varten on yhtyeen perustaja ja taiteellinen johtaja Johannes Vesterinen toimittanut nuotit nykystandardien mukaisiksi korjaten vapaasti originaalin puutteita ja virheitä. Yrjö Kari-Koskinen on tämän jälkeen huolehtinut nuottien puhtaaksikirjoituksesta ja taitosta, ja FM Jaakko Saarinen laatinut esipuheen ja selitykset. Johannes Vesterinen on lopuksi tarkistanut lopputuloksen.

Tuotosta voi halutessaan kutsua Uudeksi Porin Triviaalikoulun nuottikirjaksi tai 300-vuotisjuhlakirjaksi. Se ei ole niin sanottu kriittinen laitos tieteellisiä käyttäjiä varten. Tavoitteena on ollut käytännöllinen julkaisu, josta musiikkia olisi helppo laulaa ja soittaa. Alkuperäisiä nuottikirjoja säilytetään edelleen Porin Lyseon lukion kirjastossa, ja niiden yksityiskohdista kiinnostunut voi nykyisin helposti tutustua alkuperäisten nuottikirjojen digitoituun julkaisuun (Viergutz 2009).

Vaikka tavoitteena on käyttäjälähtöisyys, uuden julkaisun toimitustyö perustuu kokonaan historialliseen aineistoon, toisin kuin esim. Heikki Klemetin vapaat Piae Cantiones -sovitukset. Työ on tehty alkuperäistä kunnioittaen siten, että olemuksellisia muutoksia musiikille ei ole tehty; ei varsinaista sovitustyötäkään lukuunottamatta viimeiseksi liitettyä näytteenomaista sovitusta.

Huomattava osa toimituksellisesta vaivannäöstä on liittynyt käsin kirjoitettujen latinan- ja ruotsinkielisten laulutekstien tulkitsemiseen ja tavuttamiseen nuottikuvaan istuvaksi. Suhteessa alkuperäiseen nuottikuvaan on lisäksi tehty muun muassa seuraavia muutoksia:

\begin{enumerate}
\item Siirtymä erillisistä äänikirjoista yhteen partituuriin, jossa kaikki stemmat näkyvät rinnakkain. 
\item Nuottiarvoja on harkinnan mukaan puolitettu ja tahtilajeja vaihdettu vastaamaan nykykäyttäjälle tuttua notaatiotapaa
\item Tahtiviivoja on lisätty ja johdonmukaistettu vastaamaan nykykonventioita.
\item Alkuperäiset nuottiavaimet on vaihdettu nykyisen sekakuorostandardin mukaiseksi. Altus on kirjoitettu joko diskanttiavaimella tai oktaavidiskanttiavaimelle ("sekakuoron tenoriavain") sen mukaan, onko se luontevammin naisalton vai miestenorin laulettavissa.
\item Kaikki korkeille avaimille kirjoitetut sävellykset on transponoitu kvartilla tai kvintillä alas historiallisen normaalikäytännön mukaisesti. Näissä on otsikon alle lisätty "alla quarta bassa", joka ei siis tarkoita, että niitä tulisi edelleentransponoida, vaan että tämä transponointi on jo tehty puhtaaksikirjoituksen yhteydessä.
\item Ilmeiset virheet ja epäjohdonmukaisuudet on korjattu käyttäen muita lähteitä tai, niiden puuttuessa, omaa harkintaa. Häiritseväksi koetut äänenkuljetukselliset kömpelyydet, muun muassa rinnakkaiset kvintit ja oktaavit, on hiljaisesti oiottu.
\item Kolmeen kappaleeseen on lisätty Hemminki Maskulaisen suomenkieliset käännökset: Ainoan Jumalan corkeudes (Allenaste Gud i himmelrik), Äänel caunist monen mutcain (Ætas carmen melodiæ), Tuiman talven taucomast (Cedit hyems eminus) sekä Ain iloidca, ain riemuidca (Jugundare jugiter).
\end{enumerate}

Toimitustyön perusaineistona olivat Helsingin yliopiston kirjaston (nykyisin Kansalliskirjasto) mikrofilmikopiot. Suurena apuna on ollut Jyväskylän yliopiston facsimile-julkaisu (Viergutz 2009). Toissijaisina lähteinä on käytetty mm. seuraavia: \textit{Florilegium Portense} I ja II (Bodenschatz 1618, 1621), \textit{Psalm-singende Und Lobgesäng-spielende Kirche Christi In Des H. Röm. Reichs} (Helmhack 1704), \textit{Wanhain Suomen maan pijspain ja Kircon Esimiesten Latinan kielised laulud} (Hemminki Maskulainen 1616), \textit{Piae Cantiones} (Rutha 1625, myös Mats Lillhannuksen uusi editio), \textit{Then Svenska Psalmboken} (1695), \textit{Uusi Suomenkielinen Wirsi-Kirja} (1701) ja \textit{Viridarium music sacrum} (Friderici 1625). Erityiskiitokset Smålands musikarkivin avusta \textit{Låt oss liufliga siunga} -kappaleen vanhan ruotsinkielisen kaunokirjoitustekstin tulkitsemisessa.

Julkaisu on painettu omakustanteena ja on saatavilla myös verkosta osoitteessa \textit{\href{https://triviaalikoulu.sonusborealis.fi}{triviaalikoulu.sonusborealis.fi}}. Verkkoversiota pyritään päivittämään jatkuvasti ja sieltä voi ladata myös yksittäisten kappaleiden nuotteja ja yksittäisiä stemmoja. Julkaisua voi lupaa pyytämättä kopioida, muokata, levittää ja esittää, mukaan lukien kaupalliset tarkoitukset. Nuottijulkaisumme tekijänoikeudet on luovutettu yhteiseksi hyväksi CC0-lisenssin mukaisesti. Kaiken alkuperäisen musiikin tekijänoikeudet ovat rauenneet.

\section*{Lopuksi}

Suurin osa Porin Triviaalikoulun nuottikirjojen musiikista on edelleen siistimättä ja puhtaaksikirjoittamatta. Toivomme jatkossa voivamme julkaista lisää nuottikirjojen musiikkia tähän samaan tapaan puhtaaksikirjoitettuna, mahdollisesti myös sovellettuina transkriptioina. Mahdolliseen lisäosaan kuuluisivat myös tekstien käännökset, kappaleitten taustatiedot ja hakemisto.

Toimitustyö muun työn ohessa on kuitenkin hidasta. Katsoimme ajankohdan oikeaksi julkaista valmiina oleva valikoima nyt, alkuperäisten kirjojen täyttäessä tänä vuonna 300 vuotta.

Julkaisulla haluamme omalta osaltamme nostaa Porin Triviaalikoulun nuottikirjojen sisältämää vanhaa musiikkia esiin ja edistää tämän musiikin esittämistä ja jalostumista uusiksi tulkinnoiksi. Soiton ja laulun iloa!

Imatralla, Helsingissä ja Sääksmäessä 9.11.2025, \\
Johannes Vesterinen, Yrjö Kari-Koskinen ja Jaakko Saarinen
