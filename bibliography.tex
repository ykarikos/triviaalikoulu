\chapter*{Lähdeluettelo}

Anon. 1701: Uusi Suomenkielinen Wirsi-Kirja – Johann Winter. Turku. \\
% lähteenä kappaleen \emph{Ainoan Jumalan corkeudes} suomenkielisessä sanoituksessa \\
Translitterointi: \url{http://kaino.kotus.fi/korpus/vks/meta/virret/wk1701_rdf.xml} (tarkistettu 30.11.2025).

Anon. 1725: Porin triviaalikoulun äänikirjat 1: Discantus – JYX/Nuottijulkaisut. \\
Verkkojulkaisu: \url{https://urn.fi/URN:NBN:fi:jyu-200809185749} (tarkistettu 30.11.2025).

Anon. 1725: Porin triviaalikoulun äänikirjat 2: Altus – JYX/Nuottijulkaisut. \\
Verkkojulkaisu: \url{https://urn.fi/URN:NBN:fi:jyu-200809185750} (tarkistettu 30.11.2025).

Anon. 1725: Porin triviaalikoulun äänikirjat 3: Tenor – JYX/Nuottijulkaisut. \\
Verkkojulkaisu: \url{https://urn.fi/URN:NBN:fi:jyu-200809185751} (tarkistettu 30.11.2025).

Anon. 1725: Porin triviaalikoulun äänikirjat 4: Bassus – JYX/Nuottijulkaisut. \\
Verkkojulkaisu: \url{https://urn.fi/URN:NBN:fi:jyu-200809185752} (tarkistettu 30.11.2025).

Anon. 1760: Then Svenska Psalmboken [1695] – Kongl. Tryckeriet. Nyköping. \\
% lähteenä kappaleen \emph{Allenaste Gud i himmelrijk} ruotsinkielisessä sanoituksessa \\
Digitoituna: \url{https://books.google.fi/books?id=7upUAAAAcAAJ} (tarkistettu 30.11.2025).

Bodenschatz, Erhard 1618: Florilegium Portense – Abraham Lamberg \& Caspar Closemann. Lipsia [Leipzig]. \\
% lähteenä kappaleissa \emph{Repleatur os meum} ja \emph{Alleluja in resurrectione tua Christe} \\
Digitoituna: \url{https://imslp.org/wiki/Florilegium\_Portense\_I\_(Bodenschatz%2C\_Erhard)} (tarkistettu 30.11.2025).

Bodenschatz, Erhard 1621: Florilegii Musici Portensis Pars Altera – Abraham Lamberg \& Caspar Closemann. Lipsia [Leipzig]. \\
% lähteenä kappaleen \emph{Anima mea expectat Dominum} tavutuksessa \\
Digitoituna:
\url{https://imslp.org/wiki/Florilegii\_Musicii\_Portensis\_II\_(Bodenschatz%2C\_Erhard)} (tarkistettu 30.11.2025).

Friderici, Daniel 1625: Viridarium musicum sacrum, sive Cantiones Sacrae, quaternis et quinis vocibus – Sumptib: Johannis Hallervordij. Rostock. \\
% lähteenä kappaleissa \emph{Lobt Gott ihr Christen alle} ja \emph{Mens confisa Deo} \\
Digitoituna: \url{https://books.google.fi/books?id=vGWcIU4BfHgC} (tarkistettu 30.11.2025).

Haapanen, Toivo 1926: Kirkko- ja koululaulun muistomerkkejä vanhan Satakunnan alueelta – s. 101–118 teoksessa Satakunta: kotitutkimuksia VI. Satakuntalainen Osakunta. Helsinki \& Porvoo.

Haynes, Bruce 2002: A History of Performing Pitch: The Story of 'A' – Scarecrow Press. Lanham, MD.

Helmhack, Erasmus 1704: Windsheimische Gesang-Buch – Erasmus Helmhack. Windsheim. \\
% lähteenä kappaleen \emph{Lobt Gott ihr Christen alle} sanoituksessa \\
Digitoituna: \url{https://books.google.fi/books?id=\_IbEDNe32scC} (tarkistettu 30.11.2025).

Hemming, Mascun Kirckoherra [Hemminki Maskulainen] 1616: Wanhain Suomen maan pijspain ja Kircon Esimiesten Latinan kielised laulud – Ignati Meurer. Stockholmisa. \\
% lähteenä kappaleiden \emph{Jucundare jugiter}, \emph{Cedit hyems eminus} ja \emph{Ætas carmen melodiæ} suomenkielisissä sanoituksissa \\
Digitoituna: \url{https://www.doria.fi/handle/10024/59053} (tarkistettu 30.11.2025). \\
Translitteroituna: \url{http://kaino.kotus.fi/korpus/vks/meta/virret/hemm1616_rdf.xml} (tarkistettu 30.11.2025).

Klemetti, Heikki 1908: Musiikkihistoriallinen muistomerkki – Säveletär 13–14: 137–139.

Lillhannus, Mats (toim.) 2010: Piae Cantiones – Verkkojulkaisu: \url{http://www.lillhannus.net/piae-cantiones/} (tarkistettu 30.11.2025).
%käytetty lisälähteenä kappaleissa \emph{Jucundare jugiter}- ja \emph{Ætas carmen melodiæ} \\

Parrott, Andrew 2015: A brief anatomy of choirs, c.1470–1770 – s. 16–45 teoksessa Composers’ Intentions: Lost Traditions of Musical Performance. Boydell \& Brewer. Suffolk.

Praetorius, Michael 1619: Syntagmatis Musicii Tomus Tertius – Elias Holwein. Wolfenbüttel. \\
Digitoituna:
\url{https://imslp.org/wiki/Syntagma_Musicum_(Praetorius,_Michael)} (tarkistettu 30.11.2025).

Rutha, Theodoricus Petri et al. 1625: Cantiones piae et antiquae veterum episcorum et pastorum – Greifswald. \\
Digitoituna: \url{https://urn.fi/URN:NBN:fi:jyu-200806035390} (tarkistettu 30.11.2025).

Viergutz, Gudrun 2005: Beiträge zur Geschichte des Musikunterrichts an den Gelehrtenschulen der östlichen Ostseeregion im 16. und 17. Jahrhundert – Väitöskirja, Jyväskylän Yliopisto. Jyväskylä. \\
Verkkojulkaisu: \url{https://urn.fi/URN:ISBN:978-951-39-5165-8} (tarkistettu 30.11.2025).

Viergutz, Gudrun 2009: Porin triviaalikoulun ääni- eli stemmakirjat - JYX/Nuottijulkaisut. – Jyväskylän yliopiston kirjaston tiedotuslehti Verkkomakasiini. Julkaistu 2009-01-26. \\
Verkkojulkaisu: \url{https://urn.fi/URN:NBN:fi:jyu-20091261040} (tarkistettu 30.11.2025).