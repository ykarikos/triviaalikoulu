\chapter*{Lähdeluettelo}

Anon. 1701: Uusi Suomenkielinen Wirsi-Kirja – Johann Winter. Turku \\
lähteenä kappaleen \emph{Ainoan Jumalan corkeudes} suomenkielisessä sanoituksessa \\
Translitterointi: \url{http://kaino.kotus.fi/korpus/vks/meta/virret/wk1701_rdf.xml}

Anon. 1725: Porin triviaalikoulun äänikirjat 1: Discantus, Porin Lyseon kirjasto. \\
Digitoituna: \url{https://urn.fi/URN:NBN:fi:jyu-200809185749}

Anon. 1725: Porin triviaalikoulun äänikirjat 2: Altus, Porin Lyseon kirjasto. \\
Digitoituna: \url{https://urn.fi/URN:NBN:fi:jyu-200809185750}

Anon. 1725: Porin triviaalikoulun äänikirjat 3: Tenor, Porin Lyseon kirjasto. \\
Digitoituna: \url{https://urn.fi/URN:NBN:fi:jyu-200809185751}

Anon. 1725: Porin triviaalikoulun äänikirjat 4: Bassus, Porin Lyseon kirjasto. \\
Digitoituna: \url{https://urn.fi/URN:NBN:fi:jyu-200809185752}

Anon. 1760: Then Svenska Psalmboken (1695) – Kongl. Tryckeriet. Nyköping \\
lähteenä kappaleen \emph{Allenaste Gud i himmelrijk} ruotsinkielisessä sanoituksessa \\
Digitoituna: \url{https://books.google.fi/books?id=7upUAAAAcAAJ}

Bodenschatz, Erhard 1618: Florilegium Portense I – Abraham Lamberg \& Caspar Closemann. Leipzig \\
lähteenä kappaleissa \emph{Repleatur os meum} ja \emph{Alleluja in resurrectione tua Christe} \\
Digitoituna: \url{https://imslp.org/wiki/Florilegium\_Portense\_I\_(Bodenschatz%2C\_Erhard)}

Bodenschatz, Erhard 1621: Florilegii Musicii Portensis II – Abraham Lamberg \& Caspar Closemann. Leipzig \\
lähteenä kappaleen \emph{Anima mea expectat Dominum} tavutuksessa \\
Digitoituna: \url{https://imslp.org/wiki/Florilegii\_Musicii\_Portensis\_II\_(Bodenschatz%2C\_Erhard)}

Friderici, Daniel 1625: Viridarium music sacrum, sive Cantiones Sacrae, quaternis et quinis vocibus – Johann Richels Erben. Rostock \\
lähteenä kappaleissa \emph{Lobt Gott ihr Christen alle} ja \emph{Mens confisa Deo} \\
Digitoituna: \url{https://books.google.fi/books?id=vGWcIU4BfHgC}

Haapanen, Toivo 1926: Kirkko- ja koululaulun muistomerkkejä vanhan Satakunnan alueelta – s. 101–118 teoksessa Satakunta: kotitutkimuksia VI. Satakuntalainen Osakunta. Helsinki \& Porvoo.

Haynes, Bruce 2002: A History of Performing Pitch: The Story of 'A' – Scarecrow Press. Lanham, MD.

Helmhack, Erasmus 1704: Windsheimische Gesang-Buch – Erasmus Helmhack. Windsheim. \\
lähteenä kappaleen \emph{Lobt Gott ihr Christen alle} sanoituksessa \\
Digitoituna: \url{https://books.google.fi/books?id=\_IbEDNe32scC}

Klemetti, Heikki: Musiikkihistoriallinen muistomerkki – Säveletär 13–14: 137–139.

Lillhannus, Mats (toim.) 2010: Piae Cantiones \\
käytetty lisälähteenä kappaleissa \emph{Jucundare jugiter}- ja \emph{Ætas carmen melodiæ} \\
Verkkojulkaisuna: \url{http://www.lillhannus.net/piae-cantiones/}

Maskulainen, Hemminki 1616: Wanhain Suomen maan pijspain ja Kircon Esimiesten Latinan kielised laulud – Ignati Meurer. Tukholma \\
lähteenä kappaleiden \emph{Jucundare jugiter}, \emph{Cedit hyems eminus} ja \emph{Ætas carmen melodiæ} suomenkielisissä sanoituksissa \\
Digitoituna: \url{https://www.doria.fi/handle/10024/59053} \\
Translitteroituna: \url{http://kaino.kotus.fi/korpus/vks/meta/virret/hemm1616_rdf.xml}

Parrott, Andrew 2015 (2012): A brief anatomy of choirs, c.1470–1770 – s. 16–45 teoksessa Composers' Intentions: Lost Traditions of Musical Performance. Boydell \& Brewer. Suffolk.

Praetorius, Michael 1619: Syntagmatis Musicii Tomus Tertius – Elias Holwein. Wolfenbüttel.

Rutha, Theodoricus Petri 1625: Cantiones piae et antiquae veterum episcorum et pastorum. Greifswald \\
Digitoituna: \url{https://urn.fi/URN:NBN:fi:jyu-200806035390}

Viergutz, Gudrun 2005: Beiträge zur Geschichte des Musikunterrichts an den Gelehrtenschulen der östlichen Ostseeregion im 16. und 17. Jahrhundert – Väitöskirja, Jyväskylän Yliopisto. Jyväskylä. \\
Verkkojulkaisuna: \url{https://urn.fi/URN:ISBN:978-951-39-5165-8}

Viergutz, Gudrun 2009: Porin triviaalikoulun ääni- eli stemmakirjat - JYX/Nuottijulkaisut. Jyväskylä. \\
Verkkojulkaisuna: \url{https://urn.fi/URN:NBN:fi:jyu-20091261040}