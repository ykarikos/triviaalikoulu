\chapter*{Lähdeluettelo}

Bodenschatz, Erhard 1618: Florilegium Portense I. Leipzig \\
lähteenä \emph{Repleatur os meum} ja \emph{Alleluja in resurrectione tua Christe} -kappaleissa \\
\url{https://imslp.org/wiki/Florilegium\_Portense\_I\_(Bodenschatz%2C\_Erhard)}

Bodenschatz, Erhard 1621: Florilegii Musicii Portensis II. Leipzig \\
lähteenä \emph{Anima mea expectat Dominum} -kappaleen tavutuksessa \\
\url{https://imslp.org/wiki/Florilegii\_Musicii\_Portensis\_II\_(Bodenschatz%2C\_Erhard)}

Helmhack, Erasmus 1704: Psalm-singende Und Lobgesäng-spielende Kirche Christi In Des H. Röm. Reichs \\
lähteenä \emph{Lobt Gott ihr Christen alle} -kappaleen sanoituksessa \\
\url{https://books.google.fi/books?id=\_IbEDNe32scC}

Maskulainen, Hemminki 1616: Wanhain Suomen maan pijspain ja Kircon Esimiesten Latinan kielised laulud. Tukholma \\
lähteenä \emph{Jucundare jugiter}- ja \emph{Ætas carmen melodiæ} -kappaleiden suomenkielisissä sanoituksissa \\
\url{https://www.doria.fi/handle/10024/59053} \\
\url{http://kaino.kotus.fi/korpus/vks/meta/virret/hemm1616_rdf.xml}

Porin triviaalikoulun ääni- eli stemmakirjat, 1725. Pori \\
JYX/Nuottijulkaisut, Gudrun Viergutz \\
\url{https://jyx.jyu.fi/handle/123456789/19464}

Rutha, Theodoricus Petri 1625: Piae Cantiones. Greifswald \\
Mats Lillhannus edition 2010, käytetty lisälähteenä \emph{Jucundare jugiter}- ja \emph{Ætas carmen melodiæ} -kappaleiden nuotinnoksessa \\
\url{http://www.lillhannus.net/piae-cantiones/}

Then Svenska Psalmboken, 1695. Nyköping \\
lähteenä \emph{Allenaste Gud i himmelrijk} -kappaleen ruotsinkielisessä sanoituksessa \\
\url{https://books.google.fi/books?id=7upUAAAAcAAJ}

Uusi Suomenkielinen Wirsi-Kirja, 1701. Turku \\
lähteenä \emph{Ainoan Jumalan corkeudes} -kappaleen suomenkielisessä sanoituksessa \\
\url{http://kaino.kotus.fi/korpus/vks/meta/virret/wk1701_rdf.xml}
