\chapter*{Preface in English}

The four hand-copied music partbooks of Pori Trivial School are a rare surviving example of north European school music. In this case, school music means the repertoire of late 16th and early 17th century vocal polyphony prominent well into 18th century in the schools of the Baltic Sea region. The 1725 partbooks of Pori contain a representative selection of church music in 3, 4, 5, 6, and 8 parts, varying from simple rustic settings and chorale harmonisations to sumptuous double choir motets. Composer names are almost entirely absent, but many of the works can be found with composer attributions elsewhere, in widespread 17th century Lutheran music publications. The texts are mostly in Latin, but some are in Swedish, German, and Finnish.

The music is set for typical 16th century vocal scoring of (dis)cantus, altus, tenor and bassus, roughly corresponding to modern mezzo-soprano (boy treble or countertenor), high tenor, baritone and bass. It should be noted, however, that instrumental doubling or takeover of parts was a normal practice of the time. This kind of music was also performed fully instrumentally by Stadtpfeifern and other instrumental ensembles, and frequently intabulated for keyboard instruments and plucked strings. Thus, a rich variety of options is available for historically-informed performers. The music also suits well enough for modern mixed choirs, perhaps with a slight pitch adjustment.

The partbooks have until now been only available in their original format: separate manuscript parts, inconvenient for modern performers. This volume is the first modern edition of Pori Trivial School music books. The transcriptions published here were originally prepared by Johannes Vesterinen for the use of the Imatra-based early music ensemble Sonus Borealis. The ensemble has performed music from the partbooks since 2005, and also recorded a selection in 2008. Digital layout was done by Yrjö Kari-Koskinen, and the preface was added by Jaakko Saarinen, who also helped tracking down some of the composers. Thanks are due to late Dr. \mbox{Gudrun} Viergutz for sharing her scholarship, and the staff of Smålands Musikarkiv in deciphering some Swedish handwriting.

Our publication is not a scholarly, critical edition, but rather was prepared for practical use. The music has been transferred from partbook format into a full score. Old clefs have been replaced with modern standard mixed-choir clefs. Original written pitch has been retained, apart from pieces in high clefs (chiavi alti, chiavetti), which have been transposed down a fourth according to the standard 16th and 17th century convention. Obvious errors have been corrected. Original ortography of texts has been retained, but an alternative, 17th century Finnish translation has been added when available from other sources. In a potential second volume, we hope to include not only more of the music but also more information on the individual pieces, translations of the sung texts, and an index.

Copyright on the original music has expired. The copyright of this edition has been released for common good in accordance with Creative Commons licence CC0. The whole edition, as well as the individual sheet music for single songs and parts, is also available online in PDF format at \textit{\href{https://triviaalikoulu.sonusborealis.fi}{triviaalikoulu.sonusborealis.fi}}